% Options for packages loaded elsewhere
\PassOptionsToPackage{unicode}{hyperref}
\PassOptionsToPackage{hyphens}{url}
%
\documentclass[
]{book}
\usepackage{lmodern}
\usepackage{amssymb,amsmath}
\usepackage{ifxetex,ifluatex}
\ifnum 0\ifxetex 1\fi\ifluatex 1\fi=0 % if pdftex
  \usepackage[T1]{fontenc}
  \usepackage[utf8]{inputenc}
  \usepackage{textcomp} % provide euro and other symbols
\else % if luatex or xetex
  \usepackage{unicode-math}
  \defaultfontfeatures{Scale=MatchLowercase}
  \defaultfontfeatures[\rmfamily]{Ligatures=TeX,Scale=1}
\fi
% Use upquote if available, for straight quotes in verbatim environments
\IfFileExists{upquote.sty}{\usepackage{upquote}}{}
\IfFileExists{microtype.sty}{% use microtype if available
  \usepackage[]{microtype}
  \UseMicrotypeSet[protrusion]{basicmath} % disable protrusion for tt fonts
}{}
\makeatletter
\@ifundefined{KOMAClassName}{% if non-KOMA class
  \IfFileExists{parskip.sty}{%
    \usepackage{parskip}
  }{% else
    \setlength{\parindent}{0pt}
    \setlength{\parskip}{6pt plus 2pt minus 1pt}}
}{% if KOMA class
  \KOMAoptions{parskip=half}}
\makeatother
\usepackage{xcolor}
\IfFileExists{xurl.sty}{\usepackage{xurl}}{} % add URL line breaks if available
\IfFileExists{bookmark.sty}{\usepackage{bookmark}}{\usepackage{hyperref}}
\hypersetup{
  pdftitle={C C++ İnşa Sistemleri ve Paket Yöneticileri},
  pdfauthor={Mühendis Köyü},
  hidelinks,
  pdfcreator={LaTeX via pandoc}}
\urlstyle{same} % disable monospaced font for URLs
\usepackage{longtable,booktabs}
% Correct order of tables after \paragraph or \subparagraph
\usepackage{etoolbox}
\makeatletter
\patchcmd\longtable{\par}{\if@noskipsec\mbox{}\fi\par}{}{}
\makeatother
% Allow footnotes in longtable head/foot
\IfFileExists{footnotehyper.sty}{\usepackage{footnotehyper}}{\usepackage{footnote}}
\makesavenoteenv{longtable}
\usepackage{graphicx,grffile}
\makeatletter
\def\maxwidth{\ifdim\Gin@nat@width>\linewidth\linewidth\else\Gin@nat@width\fi}
\def\maxheight{\ifdim\Gin@nat@height>\textheight\textheight\else\Gin@nat@height\fi}
\makeatother
% Scale images if necessary, so that they will not overflow the page
% margins by default, and it is still possible to overwrite the defaults
% using explicit options in \includegraphics[width, height, ...]{}
\setkeys{Gin}{width=\maxwidth,height=\maxheight,keepaspectratio}
% Set default figure placement to htbp
\makeatletter
\def\fps@figure{htbp}
\makeatother
\setlength{\emergencystretch}{3em} % prevent overfull lines
\providecommand{\tightlist}{%
  \setlength{\itemsep}{0pt}\setlength{\parskip}{0pt}}
\setcounter{secnumdepth}{5}
\usepackage{booktabs}
\usepackage[]{natbib}
\bibliographystyle{apalike}

\title{C C++ İnşa Sistemleri ve Paket Yöneticileri}
\author{Mühendis Köyü}
\date{2020-06-18}

\begin{document}
\maketitle

{
\setcounter{tocdepth}{1}
\tableofcontents
}
\hypertarget{bir-tutam-yazux131}{%
\chapter*{Bir Tutam Yazı}\label{bir-tutam-yazux131}}
\addcontentsline{toc}{chapter}{Bir Tutam Yazı}

\hypertarget{uxf6nsuxf6z}{%
\section*{Önsöz}\label{uxf6nsuxf6z}}
\addcontentsline{toc}{section}{Önsöz}

C/C++ yazılım geliştirme sadece bu dil ile kod yazmayı kapsamıyor. \emph{IDE} (\textbf{Integrated Development Environment -- Tümleşik Geliştirme Ortamı}) haricincde bir editör ve derleme/bağlama/vb. işlemler için gerekli programları çoğu aynı, bazı yerleri değişen uzun paremetrelerle kullanmak durumunda kalırız. \emph{IDE} kullanımı ise bizi \emph{IDE}'ye bağımlı kıldığı gibi duruma göre işletim sistemine de bağımlı kılabilir. Bunun yanı sıra C/C++ projelerinde bu dillerde \emph{dâhili} (\textbf{builtin}) olarak gelmeyen bir çok kütüphane kullanımı mevcuttur. Peki birden fazla farklı ortamda bu kütüphanelerin o ortamlara göre varlığı, nereden indirileceği gibi problemleri geliştiriciler elle mi gerçekleştirecek?

İşte bu kitapla bu problemlere getirilen çözümlere, çözümlerin oluşturduğu yeni problemlere getirilen çözümlere ve en son da hala devam etmekte olan veya daha da yeni olan problemlere değinmeyi Mühendis Köyü olarak amaç edindik.

Türkçe kapsamlı bir kitap olmasını hedefleyerek başladığımız bu yolculuğumuzda türü ne olursa olsun bizlere ulaşacak her bir eleştiri sönük bir mumun alev almasına yardımcı olan yanan bir mumun ateşi olacaktır.

\hypertarget{katkux131da-bulunanlar}{%
\section*{Katkıda Bulunanlar}\label{katkux131da-bulunanlar}}
\addcontentsline{toc}{section}{Katkıda Bulunanlar}

Genellikle kitabın en az bir bölümünü en fazla 1 kişi üstlenecek şekilde bir strateji belirledik. Bölüm başlarında sorumlu kişinin adı geçmektedir. Bunun ile birlikte kitap oluşturulurken emek vermiş kişilerinde burada geçmesini istedik.

\begin{tabular}{l|l|l}
\hline
Ahmet Burak ÖZYURT & Enes AYDIN & Erdem GÜNEŞ\\
\hline
Melisa CEYLAN & Muhammed Erkam KOCAER & Numan Faruk AYDIN\\
\hline
Semanur AYDINLIK & Senanur PAKSOY & Süleyman Emre IŞIK\\
\hline
\end{tabular}

\hypertarget{lisans}{%
\section*{Lisans}\label{lisans}}
\addcontentsline{toc}{section}{Lisans}

Kitabın tamamı veya bir kısmı, ``kaynak gösterildiği ve değişiklik yapılmadığı'' takdirde, herhangi bir izne gerek kalmadan, her türlü ortamda çoğaltılabilir, dağıtılabilir, kullanılabilir.

\hypertarget{bu-kitap-nasux131l-geliux15ftiriliyor}{%
\section*{Bu Kitap Nasıl Geliştiriliyor}\label{bu-kitap-nasux131l-geliux15ftiriliyor}}
\addcontentsline{toc}{section}{Bu Kitap Nasıl Geliştiriliyor}

\href{https://t.me/koyumuhendis}{Mühendis Köyü telegram} grubunda bulunan kişilerce gönüllülük esasına dayalı olarak bu kitaba girişilmiştir. Kitap, \emph{R Markdown}'da \emph{bookdown} paketi kullanılarak yazılmaktadır. \href{https://github.com/MuhendisKoyu}{Mühendis Köyü} \emph{Github} organizasyonu altında bulunan \href{https://github.com/MuhendisKoyu/C-Cpp-insa-sistemleri-paket-yoneticileri}{C-Cpp-insa-sistemleri-paket-yoneticileri} reposunun \textbf{master} \emph{dalına} (\textbf{branch}) \emph{BuildingOfCCpp} ekibi tarafından yapılan değişiklikler yüklenmekte, yine ekipten biri tarafından \textbf{gh-pages} dalına ise \emph{R Markdown} olarak yazılan projenin \emph{HTML} çıktısı yüklenmektedir. Kitap geliştirilirken \emph{Trello} üzerinden paylaşım, \emph{telegram} üzerinden yardımlaşma, haberleşme ve tartışma sağlanmaktadır.

\hypertarget{giriux15f}{%
\chapter*{Giriş}\label{giriux15f}}
\addcontentsline{toc}{chapter}{Giriş}

\hypertarget{buxf6luxfcm-iuxe7eriux11fi}{%
\section*{Bölüm içeriği}\label{buxf6luxfcm-iuxe7eriux11fi}}
\addcontentsline{toc}{section}{Bölüm içeriği}

Bu kısımda önsözde kısaca bahsettiğimiz problemlerin çıkış noktaları ile C/C++ için inşa sistemleri ve paket yöneticilerine değineceğiz.

\hypertarget{ide-kullanux131mux131}{%
\section*{IDE Kullanımı}\label{ide-kullanux131mux131}}
\addcontentsline{toc}{section}{IDE Kullanımı}

\hypertarget{edituxf6r-derleyici-kullanux131mux131}{%
\section*{Editör + Derleyici Kullanımı}\label{edituxf6r-derleyici-kullanux131mux131}}
\addcontentsline{toc}{section}{Editör + Derleyici Kullanımı}

\hypertarget{cc-iuxe7in-inux15fa-sistemleri}{%
\section*{C/C++ için İnşa Sistemleri}\label{cc-iuxe7in-inux15fa-sistemleri}}
\addcontentsline{toc}{section}{C/C++ için İnşa Sistemleri}

\hypertarget{cc-iuxe7in-paket-yuxf6neticileri}{%
\section*{C/C++ için Paket Yöneticileri}\label{cc-iuxe7in-paket-yuxf6neticileri}}
\addcontentsline{toc}{section}{C/C++ için Paket Yöneticileri}

\hypertarget{part-inux15fa-sistemleri}{%
\part{İnşa Sistemleri}\label{part-inux15fa-sistemleri}}

\hypertarget{makefile}{%
\chapter*{Makefile}\label{makefile}}
\addcontentsline{toc}{chapter}{Makefile}

\hypertarget{buxf6luxfcm-iuxe7eriux11fi-1}{%
\section*{Bölüm içeriği}\label{buxf6luxfcm-iuxe7eriux11fi-1}}
\addcontentsline{toc}{section}{Bölüm içeriği}

Makefile kullanımı, GNU Make, BSD Make and NMake varyasyonlarına ve arasındaki farklara değinilecektir.

\hypertarget{cmake}{%
\chapter*{Cmake}\label{cmake}}
\addcontentsline{toc}{chapter}{Cmake}

\hypertarget{buxf6luxfcm-iuxe7eriux11fi-2}{%
\section*{Bölüm içeriği}\label{buxf6luxfcm-iuxe7eriux11fi-2}}
\addcontentsline{toc}{section}{Bölüm içeriği}

CMake kullanımı avantajı dezavantajı\ldots{}

\hypertarget{gnu-autotool}{%
\chapter*{GNU Autotool}\label{gnu-autotool}}
\addcontentsline{toc}{chapter}{GNU Autotool}

\hypertarget{buxf6luxfcm-iuxe7eriux11fi-3}{%
\section*{Bölüm içeriği}\label{buxf6luxfcm-iuxe7eriux11fi-3}}
\addcontentsline{toc}{section}{Bölüm içeriği}

GNU Autotool kullanımı avantajı dezavantajı\ldots{}

\hypertarget{msbuild}{%
\chapter*{Msbuild}\label{msbuild}}
\addcontentsline{toc}{chapter}{Msbuild}

\hypertarget{buxf6luxfcm-iuxe7eriux11fi-4}{%
\section*{Bölüm içeriği}\label{buxf6luxfcm-iuxe7eriux11fi-4}}
\addcontentsline{toc}{section}{Bölüm içeriği}

Msbuild kullanımı avantajı dezavantajı\ldots{}

\hypertarget{meson}{%
\chapter*{Meson}\label{meson}}
\addcontentsline{toc}{chapter}{Meson}

\hypertarget{buxf6luxfcm-iuxe7eriux11fi-5}{%
\section*{Bölüm içeriği}\label{buxf6luxfcm-iuxe7eriux11fi-5}}
\addcontentsline{toc}{section}{Bölüm içeriği}

Meson kullanımı avantajı dezavantajı\ldots{}

\hypertarget{qmake}{%
\chapter*{Qmake}\label{qmake}}
\addcontentsline{toc}{chapter}{Qmake}

\hypertarget{buxf6luxfcm-iuxe7eriux11fi-6}{%
\section*{Bölüm içeriği}\label{buxf6luxfcm-iuxe7eriux11fi-6}}
\addcontentsline{toc}{section}{Bölüm içeriği}

Qmake kullanımı avantajı dezavantajı\ldots{}

\hypertarget{kbuild}{%
\chapter*{Kbuild}\label{kbuild}}
\addcontentsline{toc}{chapter}{Kbuild}

\hypertarget{buxf6luxfcm-iuxe7eriux11fi-7}{%
\section*{Bölüm içeriği}\label{buxf6luxfcm-iuxe7eriux11fi-7}}
\addcontentsline{toc}{section}{Bölüm içeriği}

Kbuild kullanımı avantajı dezavantajı\ldots{}

\hypertarget{build2}{%
\chapter*{Build2}\label{build2}}
\addcontentsline{toc}{chapter}{Build2}

\hypertarget{buxf6luxfcm-iuxe7eriux11fi-8}{%
\section*{Bölüm içeriği}\label{buxf6luxfcm-iuxe7eriux11fi-8}}
\addcontentsline{toc}{section}{Bölüm içeriği}

Build2 kullanımı avantajı dezavantajı\ldots{}

\hypertarget{xmake}{%
\chapter*{Xmake}\label{xmake}}
\addcontentsline{toc}{chapter}{Xmake}

\hypertarget{buxf6luxfcm-iuxe7eriux11fi-9}{%
\section*{Bölüm içeriği}\label{buxf6luxfcm-iuxe7eriux11fi-9}}
\addcontentsline{toc}{section}{Bölüm içeriği}

Xmake kullanımı avantajı dezavantajı\ldots{}

\hypertarget{bazel}{%
\chapter*{Bazel}\label{bazel}}
\addcontentsline{toc}{chapter}{Bazel}

\hypertarget{buxf6luxfcm-iuxe7eriux11fi-10}{%
\section*{Bölüm içeriği}\label{buxf6luxfcm-iuxe7eriux11fi-10}}
\addcontentsline{toc}{section}{Bölüm içeriği}

Bazel kullanımı avantajı dezavantajı\ldots{}

\hypertarget{part-paket-yuxf6neticileri}{%
\part{Paket Yöneticileri}\label{part-paket-yuxf6neticileri}}

\hypertarget{conan}{%
\chapter*{Conan}\label{conan}}
\addcontentsline{toc}{chapter}{Conan}

\hypertarget{buxf6luxfcm-iuxe7eriux11fi-11}{%
\section*{Bölüm içeriği}\label{buxf6luxfcm-iuxe7eriux11fi-11}}
\addcontentsline{toc}{section}{Bölüm içeriği}

Conan kullanımı avantajı dezavantajı\ldots{}

\end{document}
