% Options for packages loaded elsewhere
\PassOptionsToPackage{unicode}{hyperref}
\PassOptionsToPackage{hyphens}{url}
%
\documentclass[
]{book}
\usepackage{lmodern}
\usepackage{amssymb,amsmath}
\usepackage{ifxetex,ifluatex}
\ifnum 0\ifxetex 1\fi\ifluatex 1\fi=0 % if pdftex
  \usepackage[T1]{fontenc}
  \usepackage[utf8]{inputenc}
  \usepackage{textcomp} % provide euro and other symbols
\else % if luatex or xetex
  \usepackage{unicode-math}
  \defaultfontfeatures{Scale=MatchLowercase}
  \defaultfontfeatures[\rmfamily]{Ligatures=TeX,Scale=1}
\fi
% Use upquote if available, for straight quotes in verbatim environments
\IfFileExists{upquote.sty}{\usepackage{upquote}}{}
\IfFileExists{microtype.sty}{% use microtype if available
  \usepackage[]{microtype}
  \UseMicrotypeSet[protrusion]{basicmath} % disable protrusion for tt fonts
}{}
\makeatletter
\@ifundefined{KOMAClassName}{% if non-KOMA class
  \IfFileExists{parskip.sty}{%
    \usepackage{parskip}
  }{% else
    \setlength{\parindent}{0pt}
    \setlength{\parskip}{6pt plus 2pt minus 1pt}}
}{% if KOMA class
  \KOMAoptions{parskip=half}}
\makeatother
\usepackage{xcolor}
\IfFileExists{xurl.sty}{\usepackage{xurl}}{} % add URL line breaks if available
\IfFileExists{bookmark.sty}{\usepackage{bookmark}}{\usepackage{hyperref}}
\hypersetup{
  pdftitle={C/C++ Geliştirme Ortamları},
  pdfauthor={Mühendis Köyü},
  hidelinks,
  pdfcreator={LaTeX via pandoc}}
\urlstyle{same} % disable monospaced font for URLs
\usepackage{color}
\usepackage{fancyvrb}
\newcommand{\VerbBar}{|}
\newcommand{\VERB}{\Verb[commandchars=\\\{\}]}
\DefineVerbatimEnvironment{Highlighting}{Verbatim}{commandchars=\\\{\}}
% Add ',fontsize=\small' for more characters per line
\usepackage{framed}
\definecolor{shadecolor}{RGB}{248,248,248}
\newenvironment{Shaded}{\begin{snugshade}}{\end{snugshade}}
\newcommand{\AlertTok}[1]{\textcolor[rgb]{0.94,0.16,0.16}{#1}}
\newcommand{\AnnotationTok}[1]{\textcolor[rgb]{0.56,0.35,0.01}{\textbf{\textit{#1}}}}
\newcommand{\AttributeTok}[1]{\textcolor[rgb]{0.77,0.63,0.00}{#1}}
\newcommand{\BaseNTok}[1]{\textcolor[rgb]{0.00,0.00,0.81}{#1}}
\newcommand{\BuiltInTok}[1]{#1}
\newcommand{\CharTok}[1]{\textcolor[rgb]{0.31,0.60,0.02}{#1}}
\newcommand{\CommentTok}[1]{\textcolor[rgb]{0.56,0.35,0.01}{\textit{#1}}}
\newcommand{\CommentVarTok}[1]{\textcolor[rgb]{0.56,0.35,0.01}{\textbf{\textit{#1}}}}
\newcommand{\ConstantTok}[1]{\textcolor[rgb]{0.00,0.00,0.00}{#1}}
\newcommand{\ControlFlowTok}[1]{\textcolor[rgb]{0.13,0.29,0.53}{\textbf{#1}}}
\newcommand{\DataTypeTok}[1]{\textcolor[rgb]{0.13,0.29,0.53}{#1}}
\newcommand{\DecValTok}[1]{\textcolor[rgb]{0.00,0.00,0.81}{#1}}
\newcommand{\DocumentationTok}[1]{\textcolor[rgb]{0.56,0.35,0.01}{\textbf{\textit{#1}}}}
\newcommand{\ErrorTok}[1]{\textcolor[rgb]{0.64,0.00,0.00}{\textbf{#1}}}
\newcommand{\ExtensionTok}[1]{#1}
\newcommand{\FloatTok}[1]{\textcolor[rgb]{0.00,0.00,0.81}{#1}}
\newcommand{\FunctionTok}[1]{\textcolor[rgb]{0.00,0.00,0.00}{#1}}
\newcommand{\ImportTok}[1]{#1}
\newcommand{\InformationTok}[1]{\textcolor[rgb]{0.56,0.35,0.01}{\textbf{\textit{#1}}}}
\newcommand{\KeywordTok}[1]{\textcolor[rgb]{0.13,0.29,0.53}{\textbf{#1}}}
\newcommand{\NormalTok}[1]{#1}
\newcommand{\OperatorTok}[1]{\textcolor[rgb]{0.81,0.36,0.00}{\textbf{#1}}}
\newcommand{\OtherTok}[1]{\textcolor[rgb]{0.56,0.35,0.01}{#1}}
\newcommand{\PreprocessorTok}[1]{\textcolor[rgb]{0.56,0.35,0.01}{\textit{#1}}}
\newcommand{\RegionMarkerTok}[1]{#1}
\newcommand{\SpecialCharTok}[1]{\textcolor[rgb]{0.00,0.00,0.00}{#1}}
\newcommand{\SpecialStringTok}[1]{\textcolor[rgb]{0.31,0.60,0.02}{#1}}
\newcommand{\StringTok}[1]{\textcolor[rgb]{0.31,0.60,0.02}{#1}}
\newcommand{\VariableTok}[1]{\textcolor[rgb]{0.00,0.00,0.00}{#1}}
\newcommand{\VerbatimStringTok}[1]{\textcolor[rgb]{0.31,0.60,0.02}{#1}}
\newcommand{\WarningTok}[1]{\textcolor[rgb]{0.56,0.35,0.01}{\textbf{\textit{#1}}}}
\usepackage{longtable,booktabs}
% Correct order of tables after \paragraph or \subparagraph
\usepackage{etoolbox}
\makeatletter
\patchcmd\longtable{\par}{\if@noskipsec\mbox{}\fi\par}{}{}
\makeatother
% Allow footnotes in longtable head/foot
\IfFileExists{footnotehyper.sty}{\usepackage{footnotehyper}}{\usepackage{footnote}}
\makesavenoteenv{longtable}
\usepackage{graphicx,grffile}
\makeatletter
\def\maxwidth{\ifdim\Gin@nat@width>\linewidth\linewidth\else\Gin@nat@width\fi}
\def\maxheight{\ifdim\Gin@nat@height>\textheight\textheight\else\Gin@nat@height\fi}
\makeatother
% Scale images if necessary, so that they will not overflow the page
% margins by default, and it is still possible to overwrite the defaults
% using explicit options in \includegraphics[width, height, ...]{}
\setkeys{Gin}{width=\maxwidth,height=\maxheight,keepaspectratio}
% Set default figure placement to htbp
\makeatletter
\def\fps@figure{htbp}
\makeatother
\setlength{\emergencystretch}{3em} % prevent overfull lines
\providecommand{\tightlist}{%
  \setlength{\itemsep}{0pt}\setlength{\parskip}{0pt}}
\setcounter{secnumdepth}{5}
\usepackage{booktabs}
\usepackage[]{natbib}
\bibliographystyle{apalike}

\title{C/C++ Geliştirme Ortamları}
\author{Mühendis Köyü}
\date{2020-07-21}

\begin{document}
\maketitle

{
\setcounter{tocdepth}{1}
\tableofcontents
}
\hypertarget{bir-tutam-yazux131}{%
\chapter{Bir Tutam Yazı}\label{bir-tutam-yazux131}}

\hypertarget{uxf6nsuxf6z}{%
\section{Önsöz}\label{uxf6nsuxf6z}}

C/C++ geliştirme ortamı sadece bu diller ile kod yazmayı kapsamıyor. \emph{IDE} (\textbf{Integrated Development Environment -- Tümleşik Geliştirme Ortamı}) ortamı dışında geliştirdiğimiz yazılımlarda derleme, bağlama vb. işlemler sırasında projelere göre farklılık gösteren uzun paremetrelerle kullanmak durumunda kalırız. \emph{IDE} kullanımı ise bizi \emph{IDE}'ye bağımlı kıldığı gibi duruma göre işletim sistemine de bağımlı kılabilir. Bunun yanı sıra C/C++ projelerinde bu dillerde \emph{dâhili} (\textbf{built-in}) olarak gelmeyen bir çok kütüphane kullanımı mevcuttur. Peki birden fazla farklı ortamda bu kütüphanelerin o ortamlara göre varlığı, nereden indirileceği gibi problemleri geliştiriciler elle mi gerçekleştirmek zorundadır?

İşte bu kitapla bahsi geçen problemler için geliştirilen çözümlere, çözümlerin oluşturduğu yeni problemlere getirilen çözümlere ve en son da hala devam etmekte olan veya daha da yeni olan problemlere değinmeyi Mühendis Köyü olarak amaç edindik.

Kapsamlı bir Türkçe kaynak olmasını hedefleyerek başladığımız bu yolculuğumuzda türü ne olursa olsun bizlere ulaşacak her bir eleştiri sönük bir mumun alev almasına yardımcı olan yanan bir mumun ateşi olacaktır.

\hypertarget{katkux131da-bulunanlar}{%
\section{Katkıda Bulunanlar}\label{katkux131da-bulunanlar}}

Genellikle kitabın en az bir bölümünü en fazla 1 kişi üstlenecek şekilde bir strateji belirledik. Bölüm başlarında sorumlu kişinin adı geçmektedir. Bununla birlikte kitap oluşturulurken emek vermiş kişilerinde burada geçmesini istedik.

\begin{tabular}{l|l|l}
\hline
 &  & \\
\hline
Ahmet B. ÖZYURT & Enes AYDIN & Erdem GÜNEŞ\\
\hline
Melisa CEYLAN & Muhammed E. KOCAER & Numan F. AYDIN\\
\hline
Semanur AYDINLIK & Senanur PAKSOY & Süleyman E. IŞIK\\
\hline
\end{tabular}

\hypertarget{lisans}{%
\section{Lisans}\label{lisans}}

Kitabın tamamı veya bir kısmı, ``kaynak gösterildiği ve değişiklik yapılmadığı'' takdirde, herhangi bir izne gerek kalmadan, her türlü ortamda çoğaltılabilir, dağıtılabilir, kullanılabilir.

\hypertarget{bu-kitap-nasux131l-geliux15ftiriliyor}{%
\section{Bu Kitap Nasıl Geliştiriliyor}\label{bu-kitap-nasux131l-geliux15ftiriliyor}}

\href{https://t.me/koyumuhendis}{Mühendis Köyü telegram} grubunda bulunan kişilerce gönüllülük esasına dayalı olarak bu kitaba girişilmiştir. Kitap, \emph{R Markdown}'da \emph{bookdown} paketi kullanılarak yazılmaktadır. \href{https://github.com/MuhendisKoyu}{Mühendis Köyü} \emph{Github} organizasyonu altında bulunan \href{https://github.com/MuhendisKoyu/C-Cpp-gelistirme-ortamlari}{C-Cpp-gelistirme-ortamlari} reposunun \textbf{master} \emph{dalına} (\textbf{branch}) \emph{CGOY} (\textbf{C Geliştirme Ortamı Yazarları}) ekibi tarafından yapılan değişiklikler yüklenmekte, yine ekipten biri tarafından \textbf{gh-pages} dalına ise \emph{R Markdown} olarak yazılan projenin \emph{HTML} çıktısı yüklenmektedir. Kitap geliştirilirken \emph{Trello} üzerinden paylaşım, \emph{telegram} üzerinden yardımlaşma, haberleşme ve tartışma sağlanmaktadır.

\hypertarget{giriux15f}{%
\chapter{Giriş}\label{giriux15f}}

\hypertarget{buxf6luxfcm-iuxe7eriux11fi}{%
\section{Bölüm içeriği}\label{buxf6luxfcm-iuxe7eriux11fi}}

Bu kısımda önsözde kısaca bahsettiğimiz problemlerin çıkış noktaları ile C/C++ için inşa sistemleri ve paket yöneticilerine değineceğiz.

\hypertarget{ide-kullanux131mux131}{%
\section{IDE Kullanımı}\label{ide-kullanux131mux131}}

\hypertarget{edituxf6r-derleyici-kullanux131mux131}{%
\section{Editör + Derleyici Kullanımı}\label{edituxf6r-derleyici-kullanux131mux131}}

Bir çok yazılım dilinde olduğu gibi programlarımızı IDE'ler olmadan da geliştirme imkanımız bulunmaktadır. IDE'ler güçlü bir geliştirme imkanı sunmasına rağmen yeni başlayanlar için karmaşık bir hal alabilmektedir. Aynı zamanda kaynak tüketimi editörlere göre daha yüksektir. Bu bölümde C yazılımları geliştirebileceğiniz en temel seviyeli editör olan \textbf{Vim}'e değinip, programlarınızı nasıl manuel olarak derleyebileceğinizden bahsedeceğiz.

Bu başlık altında VIM editörü hakkında anlatılacak bilgiler \href{https://missing-semester-tr.github.io/}{MIT \textbar{} The Missing Semester of Your CS Education TR}'den alınmıştır. VIM dahil diğer dersleri kaynak linkinden inceleyebilirsiniz.

\hypertarget{vim}{%
\subsection{VIM}\label{vim}}

\hypertarget{vimin-felsefesi}{%
\subsubsection{Vim'in Felsefesi}\label{vimin-felsefesi}}

Programlama yaparken zamanınızın çoğunu yazmaya değil okumaya/düzenlemeye harcarsınız. Bu yüzden Vim farklı modlara sahip bir editördür: metin eklemek veya metin işlemek için farklı modlara sahiptir. Vim programlanabilir (Vimscript ve Python gibi diğer diller ile) ve Vim'in arayüzünün kendisi bir programlama dilidir. Vim, fare kullanımından kaçınır, çünkü çok yavaştır; Vim, çok fazla hareket gerektirdiği için ok tuşlarını kullanmaktan bile kaçınır.

Sonuç olarak Vim, düşündüğünüz kadar hızlı olan bir editördür.

\textbf{Modal Düzenleme}
Vim'in tasarımı, uzun metin akışları yazmak yerine, çok sayıda programcının zamanını; okumak, gezinmek ve küçük düzenlemeler yapmak için harcandığı fikrine dayanır. Bu nedenle Vim'in birden fazla çalışma modu vardır.

\begin{itemize}
\tightlist
\item
  \textbf{Normal:} dosyanın içerisinde gezinmek ve değişiklikler yapmak için,
\item
  \textbf{Insert:} metin eklemek için,
\item
  \textbf{Replace:} metni değiştirmek için,
\item
  \textbf{Visual (Plain, Line or Block):} metin bloklarını seçmek için,
\item
  \textbf{Command-line:} bir komut çalıştırmak için
\end{itemize}

Klayve tuşlarının farklı çalışma modlarında farklı anlamları vardır. Örnek olarak, \textbf{Insert} modunda iken \texttt{x} harfine basarsak o harfi ekleyecektir ama Normal modda iken \texttt{x} harfi imlecin altındaki karakteri siler ve Visual modda ise seçili olanı siler.

Varsayılan ayarlarda Vim, o anki çalışma modunu sol altta gösterir. Başlangıç modu/varsayılan mod Normal moddur. Genellikle zamanının çoğunu Normal mod ve Insert mod arasında geçireceksin. Herhangi bir moddan Normal moda geri dönmek için \texttt{\textless{}ESC\textgreater{}} tuşuna basarak modları değiştirebilirsiniz. Normal moddan \texttt{i} ile Insert moduna, \texttt{R} ile Replace moduna, \texttt{v} ile Visual moduna, \texttt{V} ile Visual Line moduna, \texttt{\textless{}C-v\textgreater{}} ile Visual Block moduna, \texttt{:} ile Command-line moduna girebilirsin.

\hypertarget{temel-uxf6ux11feler}{%
\subsection{Temel Öğeler}\label{temel-uxf6ux11feler}}

\hypertarget{metin-ekleme}{%
\subsubsection{Metin Ekleme}\label{metin-ekleme}}

Normal modda iken Insert moduna girmek için \texttt{i} tuşuna basın. Şimdi Vim, Normal moda geri dönmek için \texttt{\textless{}ESC\textgreater{}} tuşuna basana kadar diğer metin editörleri gibi çalışır. Bu bilgi ve yukarıda açıklanan temel bilgilerle birlikte, Vim'i kullanarak dosyaları düzenlemeye başlamak için ihtiyacınız olan tek şeydir (eğer bütün zamanınızı Insert Modundan düzenleme için harcıyorsanız çok da verimli değil).

\hypertarget{command-line}{%
\subsubsection{Command-line}\label{command-line}}

Command moduna Normal modda iken \texttt{:} yazarak giriş yapabiliriz. \texttt{:} Tuşuna bastığınızda bilgisayarınızın imleci ekranın altındaki komut satırına atlayacaktır. Bu mod, dosyaları açma, kaydetme, kapatma ve \href{https://twitter.com/iamdevloper/status/435555976687923200}{Vim'den çıkış} yapma gibi birçok işleve sahiptir.

\begin{itemize}
\tightlist
\item
  \texttt{:q} çıkış (pencereyi kapatır)
\item
  \texttt{:w} kayıt (``yaz'')
\item
  \texttt{:wq} kaydet ve çık
\item
  \texttt{:e\ \{dosyanın\ adı\}} düzenlemek için dosyayı açar
\item
  \texttt{:ls} açık bufferları gösterir
\item
  \texttt{:help\ \{konu\}} yardımı açar

  \begin{itemize}
  \tightlist
  \item
    \texttt{:help\ :w} \texttt{:w} komutu için yardımı açar
  \item
    \texttt{:help\ w} \texttt{w} tuşu için yardımı açar
  \end{itemize}
\end{itemize}

Dersin devamına ve konunun detayına \href{https://missing-semester-tr.github.io/}{Missing Semester TR} kaynağından devam edebilirsiniz. Bunun yanı sıra eğer pratik yapmak istiyorsanız terminale \texttt{vimtutor} yazarak Vim ile birlikte gelen eğitimi tamamlayabilirsiniz.

\hypertarget{vimde-c-programux131-geliux15ftirme}{%
\subsection{VIM'de C Programı Geliştirme}\label{vimde-c-programux131-geliux15ftirme}}

\texttt{vim\ merhaba.c} ile merhaba isimli C dosyamızı açıp \texttt{i} harfi ile Insert moduna geçerek geleneksel uygulamamızı yazmaya başlayabiliriz.

\begin{Shaded}
\begin{Highlighting}[]
\CommentTok{\#include <stdio.h>}
\BuiltInTok{int}\NormalTok{ main()}

\NormalTok{\{}
\NormalTok{  printf(}\StringTok{"Merhaba Mühendis Köyü!"}\NormalTok{)}\OperatorTok{;}
  \ControlFlowTok{return} \DecValTok{0}\OperatorTok{;}
\NormalTok{\}}
\end{Highlighting}
\end{Shaded}

Kodu tamamladıktan sonra \texttt{\textless{}ESC\textgreater{}} tuşu ile komut moduna geçip \texttt{:wq} yazarak kaydedip çıkış yapıyoruz. Yazdığımız kodu gcc ile derlemek için (clang kullanıyorsanız gcc yerine clang yazmanız yeterlidir):

\texttt{gcc\ merhaba.c\ \_//derlenmiş\ kodu\ a.out\ olarak\ çıktı\ verir\_}
\texttt{gcc\ merhaba.c\ -o\ merhaba\ \_//derlenmiş\ kodu\ merhaba\ olarak\ çıktı\ verir\_}

Derlenen kodu çalıştırmak için \texttt{./a.out} veya \texttt{./merhaba} komutunu çalıştırmanız yeterlidir.

Editör kullanarak yazdığımız kodlarda, programımızı derlerken dahil ettiğimiz kütüphaneleri manuel olarak bağlantılamamız (kullandığımız kütüphaneyi derleyiciye belirtmemiz) gerekmektedir. Buna örnek bir kod örneği vermemiz gerekirse:

\texttt{vim\ us\_alma.c}

\begin{verbatim}
#include <stdio.h>
#include <math.h>

int main()
{
    int taban, us; 
  
    printf("Taban sayısını giriniz: ");
    scanf("%i", &taban);
  
    printf("Üs sayısını giriniz: ");
    scanf("%i", &us);

    printf("%1.f\n", pow((double)taban,(double)us));

    return 0;
}
\end{verbatim}

Yazdığımız kodu kaydedip kapattıktan sonra derleme işlemini aşağıdaki şekilde gerçekleştirirseniz derleyici bilinmeyen referans hatası verecektir.

\texttt{gcc\ us\_alma.c\ -o\ us\_alma}

Math kütüphanesini dahil ettiğimiz kodumuzu belirtmiş olduğumuz şekilde ekli kütüphaneye bağlantı vererek derlemek için derleme komutuna -lm komutunu dahil etmemiz gerekmektedir. \texttt{-l} komutu bağlantılama (linkleme) komutu olup kendisinden sonra gelen argümandaki kütüphaneyi derleme işlemine dahil eder.

\texttt{gcc\ us\_alma.c\ -o\ us\_alma\ -lm}

Bu şekilde kodumuz uygun bir biçimde derlenecektir. \texttt{./us\_alma} yazarak kodumuzu çalıştırabiliriz.

Vim ile temel seviyede C kodu yazılması ve derlenmesi bu şekilde özetlenebilir. Vim veya başka bir editör (VS Code, SublimeText vb.) kullanarak yazdığımız programları manuel olarak derlememiz gerekmektedir. Programlarımızın kapsamı genişledikçe bağlantılamamız gereken kütüphane sayısı artmakta ve bunu sürekli olarak yapmak zor bir hal almaktadır. Bu durumları hızlandırmak için ise makefile dediğimiz linkleme işlemlerini bizim için gerçekleştiren programların yazılmasına bu dokümanın ileriki bölümlerinde değineceğiz.

\hypertarget{cc-iuxe7in-inux15fa-sistemleri}{%
\section{C/C++ için İnşa Sistemleri}\label{cc-iuxe7in-inux15fa-sistemleri}}

\hypertarget{cc-iuxe7in-paket-yuxf6neticileri}{%
\section{C/C++ için Paket Yöneticileri}\label{cc-iuxe7in-paket-yuxf6neticileri}}

\hypertarget{part-inux15fa-sistemleri}{%
\chapter{(PART) İnşa Sistemleri}\label{part-inux15fa-sistemleri}}

\hypertarget{makefile}{%
\chapter{Makefile}\label{makefile}}

\hypertarget{makefile-nedir}{%
\section{Makefile Nedir}\label{makefile-nedir}}

Makefile kullanımı, GNU Make, BSD Make and NMake varyasyonlarına ve arasındaki farklara değinilecektir.

\hypertarget{makefile-kullanux131mux131}{%
\section{Makefile Kullanımı}\label{makefile-kullanux131mux131}}

Makefile kullanımı, GNU Make, BSD Make and NMake varyasyonlarına ve arasındaki farklara değinilecektir.

\hypertarget{makefile-avantajlarux131}{%
\section{Makefile Avantajları}\label{makefile-avantajlarux131}}

Makefile kullanımı, GNU Make, BSD Make and NMake varyasyonlarına ve arasındaki farklara değinilecektir.

\hypertarget{makefile-dezavantajlarux131}{%
\section{Makefile Dezavantajları}\label{makefile-dezavantajlarux131}}

Makefile kullanımı, GNU Make, BSD Make and NMake varyasyonlarına ve arasındaki farklara değinilecektir.

\hypertarget{cmake}{%
\chapter{Cmake}\label{cmake}}

\hypertarget{cmake-nedir}{%
\section{Cmake Nedir}\label{cmake-nedir}}

CMake kullanımı avantajı dezavantajı\ldots{}

\hypertarget{cmake-kullanux131mux131}{%
\section{Cmake Kullanımı}\label{cmake-kullanux131mux131}}

CMake kullanımı avantajı dezavantajı\ldots{}

\hypertarget{cmake-avantajlarux131}{%
\section{Cmake Avantajları}\label{cmake-avantajlarux131}}

CMake kullanımı avantajı dezavantajı\ldots{}

\hypertarget{cmake-dezavantajlarux131}{%
\section{Cmake Dezavantajları}\label{cmake-dezavantajlarux131}}

CMake kullanımı avantajı dezavantajı\ldots{}

\hypertarget{gnu-autotool}{%
\chapter{GNU Autotool}\label{gnu-autotool}}

\hypertarget{genel-bakux131ux15f}{%
\section{Genel Bakış}\label{genel-bakux131ux15f}}

\begin{itemize}
\item
  Autotools

  \begin{itemize}
  \tightlist
  \item
    =\textgreater\textgreater{} Autoconf, automake ve libtool için GNU Yapı Sistemi veya umbrela adı. GNU otomatik araçları birçok açık kaynaklı Linux projesi ve gömülü Linux tarafından kullanılmaktadır.
  \end{itemize}
\item
  GNU Autotools Kullanan Bazı Projeler:

  \begin{itemize}
  \tightlist
  \item
    \href{https://github.com/emacs-mirror/emacs}{GNU Emacs}
  \item
    \href{https://github.com/bminor/glibc}{GNU Glibc} (Linux'ta kullanılan GNU C Çalışma Zamanı Kütüphanesi)
  \item
    \href{https://github.com/gcc-mirror/gcc/blob/master/configure.ac}{GNU GCC Compiler}
  \item
    \href{https://gitlab.freedesktop.org/NetworkManager/NetworkManager/blob/master/configure.ac}{NetworkManager} (Linux ağ yöneticisi arka plan programı)
  \item
    \href{https://github.com/hishamhm/htop/blob/master/configure.ac}{Htop tool} - htop etkileşimli bir metin modu süreç görüntüleyicisidir.
  \item
    \href{https://github.com/nghttp2/nghttp2/blob/master/configure.ac}{Nhttp2} - (Köprü Metni Aktarım Protokolü sürüm 2'nin C'de uygulanması.)
  \item
    \href{https://github.com/mingw-deb/libffi/blob/master/configure}{LibFFI}
  \item
    \href{https://github.com/wireapp/libsodium/blob/master/configure.ac}{LibSodium}
  \end{itemize}
\item
  Çoğunlukla Unix benzeri veya POSIX benzeri işletim sistemlerinde, özellikle Linux'ta kullanılır.
\item
  Motivasyon:

  \begin{itemize}
  \tightlist
  \item
    Eski sistemler veya yazılımlarla başa çıkmak,
  \item
    Alternatif inşa sistemleri hakkında bilgi edinin,
  \item
    Mevcut açık kaynaklı projelerin bakımı,
  \item
    GNU otomatik araçlarını CMake'e taşıma olanağı,
  \item
    Gömülü Linux.
  \end{itemize}
\item
  Desteklenen IDE'ler (Entegre Geliştirme Ortamları)

  \begin{itemize}
  \tightlist
  \item
    Eclipse IDE

    \begin{itemize}
    \tightlist
    \item
      \href{https://www.eclipse.org/linuxtools/}{Linux Araçları Eklentisi}
    \item
      \href{https://wiki.eclipse.org/CDT/Autotools/User_Guide}{Eclipse CDT Autotools Kullanım Kılavuzu}
    \end{itemize}
  \item
    NetBeans

    \begin{itemize}
    \tightlist
    \item
      \href{http://plugins.netbeans.org/plugin/51647/cppgnuautotools}{CppGnuAutoTools Eklentisi}
    \end{itemize}
  \item
    QTCreator

    \begin{itemize}
    \tightlist
    \item
      \href{https://doc.qt.io/qtcreator/creator-projects-autotools.html}{Autotools Proje Yöneticisi Eklentisi}
    \end{itemize}
  \end{itemize}
\item
  Şunun için kullanılabilir:

  \begin{itemize}
  \tightlist
  \item
    Linux, MacOSX, FreeBSD, OpenBSD, NetBSD vb.
  \end{itemize}
\item
  Belgeler:

  \begin{itemize}
  \tightlist
  \item
    \href{https://www.gnu.org/software/autoconf/}{GNU Autoconf}
  \item
    \href{https://www.gnu.org/software/automake/automake.html}{GNU Automake}
  \item
    \href{https://www.gnu.org/software/libtool/libtool.html}{GNU Libtool}
  \end{itemize}
\item
  Desteklenen Programlama Dilleri

  \begin{itemize}
  \tightlist
  \item
    C
  \item
    C++
  \item
    Objective C
  \item
    Fortran
  \item
    Fortran77
  \item
    Erlang
  \end{itemize}
\item
  Terminal Belgeleri:

  \begin{itemize}
  \tightlist
  \item
    \texttt{\$\ info\ automake}
  \item
    \texttt{\$\ info\ autoconf}
  \item
    \texttt{\$\ info\ libtool}
  \end{itemize}
\item
  Ortak yapı yapılandırma olanakları

  \begin{itemize}
  \tightlist
  \item
    \texttt{\$\ ./configure}
  \item
    \texttt{\$\ make}
  \item
    \texttt{\$\ make\ install\ =\textgreater{}\ Install\ application.}
  \item
    \texttt{\$\ \ldots{}\ \ldots{}\ ..}
  \end{itemize}
\end{itemize}

\hypertarget{referans-kartux131}{%
\section{Referans Kartı}\label{referans-kartux131}}

\hypertarget{kurulum}{%
\subsection{Kurulum}\label{kurulum}}

\hypertarget{fedora-linux-daux11fux131tux131mux131na-kurulumu}{%
\paragraph{Fedora Linux Dağıtımına Kurulumu:}\label{fedora-linux-daux11fux131tux131mux131na-kurulumu}}

\texttt{\$\ sudo\ dnf\ install\ autoconf\ automake}

\hypertarget{ubuntu-veya-debian-linux-daux11fux131tux131mlarux131na-kurulumu}{%
\paragraph{Ubuntu veya Debian Linux Dağıtımlarına Kurulumu:}\label{ubuntu-veya-debian-linux-daux11fux131tux131mlarux131na-kurulumu}}

\texttt{\$\ sudo\ apt-get\ install\ autoconf\ automake}

\hypertarget{gnu-autotools-iuxe7in-gnu-make-komutlarux131}{%
\subsection{GNU Autotools için GNU Make Komutları}\label{gnu-autotools-iuxe7in-gnu-make-komutlarux131}}

\begin{verbatim}
Make Komutları -> Açıklamaları
$ make -> Uygulamayı oluştur
$ make -j4 -> 4 iş parçacığı kullanarak uygulama oluşturma

 	 
$ make check -> Testleri çalıştırın
 	 
$ make clean -> Derleme dosyalarını temizle
$ make distclean -> Oluşturulan otomatik dosyaları kaldırın.
 	 
$ make install -> / Usr / loca / bin, / etc,… için uygulamayı sisteme yükleyin.
$ make intallcheck -> Kurulumu kontrol edin
$ make uninstall -> Yükleme dosyalarını kaldırma
 	 
$ make dist -> Kaynak dağıtım tarball oluştur
$ make distcheck -> Dist yapmaya benzer, ancak derleme olup olmadığını kontrol eder, bir tarball oluşturur.
\end{verbatim}

\hypertarget{uxe7oux11fu-yapux131landux131rma-deux11fiux15fkeni}{%
\subsubsection{Çoğu Yapılandırma değişkeni}\label{uxe7oux11fu-yapux131landux131rma-deux11fiux15fkeni}}

\begin{verbatim}
Değişken -> Açıklama
CC -> C derleyici	 
CXX -> C++ derleyici	 
 	 	 
CPPFLAGS -> C ve C ++ Önişlemci bayrakları	 
CFLAGS -> C derleyicisinin bayrakları
CXXFLAGS -> C ++ derleyicisinin bayrakları 
 	 	 
LDFLAGS -> Bağlayıcı Bayrakları 
LIBS -> Bağlayıcıya aktarılan kütüphaneler
\end{verbatim}

\hypertarget{faydalux131-.configure-autotools-komut-satux131rux131-anahtarlarux131}{%
\paragraph{Faydalı ./configure (Autotools) komut satırı anahtarları}\label{faydalux131-.configure-autotools-komut-satux131rux131-anahtarlarux131}}

Bunlar, açık kaynaklı uygulamalar veya kütüphaneden kaynaktan kütüphaneler oluşturmak için kullanışlı ve yinelenen komut satırı anahtarlarıdır.

\begin{itemize}
\tightlist
\item
  ./Configure yardım kısmını göster
\end{itemize}

\texttt{\$\ ./configure\ -\/-help}

\begin{itemize}
\tightlist
\item
  Genel ./configure kullanım örneği.
\end{itemize}

\begin{verbatim}
# ---- Example 1 --------------------#
# Install to /usr, /usr/bin, /usr/include, /usr/lib, /usr/lib64 on Linux 
$ ./configure  
$ make 
$ make install 

# ---- Example 2 --------------------#
$ ./configure  --with-feature1 --without-feature2 --eanable-option1 --enable-option2=no 
$ make 
$ make install 

# ---- Example 3 --------------------#
# Install to custom location (directory)
$ ./configure --prefix=/my/custom/location  --with-feature1 \
     --without-feature2 --eanable-option1 --enable-option2=no --disable-option3

$ make -j4 # Build using 4 threads (faster)
$ make install 
\end{verbatim}

\hypertarget{kullanux131cux131lar-iuxe7in-autotools-iux15f-akux131ux15fux131}{%
\subsection{Kullanıcılar için Autotools iş akışı}\label{kullanux131cux131lar-iuxe7in-autotools-iux15f-akux131ux15fux131}}

ADIM 1: Kullanıcı ./configure inşa seçeneğini seçerek çalıştırır;

\begin{verbatim}
# Install application in default directory
$ ./configure

# Install the application in another directory 
$ ./configure --prefix=/home/user/juliuscaesar/opt
\end{verbatim}

ADIM 2: İnşa için (GNU) make çalıştırmak.

\begin{verbatim}
$ make 

# Run GNY make with 4 threads 
$ make -j4 

# Run $ make and $ make install in a single step 
$ make -j4 && sudo make install 
\end{verbatim}

ADIM 3: Kurulum.

\begin{verbatim}
$ make install 

# If permission is needed 
$ sudo make install 
\end{verbatim}

\hypertarget{geliux15ftirici-worflow}{%
\subsection{Geliştirici Worflow}\label{geliux15ftirici-worflow}}

ADIM 1:

\begin{itemize}
\tightlist
\item
  =\textgreater{} Dosya oluştur:

  \begin{itemize}
  \tightlist
  \item
    autoconf için configure.ac - M4 makro işlemcisi tarafından işlenen Bash'a (Bourne Kabuk Betiği) benzer bir dilde yazılmıştır.

    \begin{itemize}
    \tightlist
    \item
      Not: Makro invokasyonlardaki boşluklara dikkat etmek gerekir.
    \end{itemize}
  \item
    Automake için Makefile.am.
  \end{itemize}
\end{itemize}

ADIM 2:

\begin{itemize}
\tightlist
\item
  =\textgreater{} Geliştirici ./configure ve Makefile.in dosyalarını oluşturan autoreconf komutunu çalıştırır.
\end{itemize}

ADIM 3:

\begin{itemize}
\tightlist
\item
  =\textgreater{} Oluşturulan dosyaları çalıştırarak test edin:

  \begin{itemize}
  \tightlist
  \item
    \texttt{\$\ ./configure\ \&\&\ make\ \&\&\ make\ install.}
  \end{itemize}
\end{itemize}

\hypertarget{uxf6rnek-temel-gnu-autotools-projesi}{%
\section{Örnek: Temel GNU autotools projesi}\label{uxf6rnek-temel-gnu-autotools-projesi}}

\hypertarget{proje-dosyalarux131}{%
\subsection{Proje Dosyaları}\label{proje-dosyalarux131}}

Dosya: main.cpp

\begin{verbatim}
#include <iostream>

int main()
{
  std::cout << "Hello world autotools" << std::endl;
  return 0;
}
\end{verbatim}

\hypertarget{dosya-configure.ac}{%
\paragraph{Dosya: configure.ac}\label{dosya-configure.ac}}

\begin{verbatim}
#------- File: configure.ac ----------#
#-------------------------------------# 

# app => Is the file name without extensions
# 0.1 => Is the version 
AC_INIT( [app], [0.1], [maintener@email.com])

# Require at least autoconf version >= 2.68 
AC_PREREQ([2.68])  
AM_INIT_AUTOMAKE([-Wall -Wextra])

# ./configure creates a Makefile
AC_CONFIG_FILES([Makefile])

# Find and check C compiler 
AC_PROG_CC
# Find and check a C++ compiler 
AC_PROG_CXX 
AC_OUTPUT 
\end{verbatim}

\hypertarget{dosya-makefile.am}{%
\paragraph{Dosya: Makefile.am}\label{dosya-makefile.am}}

\begin{verbatim}
# Contains all executable targets 
bin_PROGRAMS = app

# Sources for the executable targets 
app_SOURCES  = app.cpp 
\end{verbatim}

\hypertarget{dosya-clean.sh}{%
\paragraph{Dosya: clean.sh}\label{dosya-clean.sh}}

\begin{itemize}
\tightlist
\item
  Geçerli dizini temizlemek için kullanılan yardımcı bash betiği.
\end{itemize}

\begin{verbatim}
#!/usr/bin/env sh 
rm -v configure config.log config.status install-sh missing aclocal.m4 compile depcomp 
rm -vrf autom4te.cache
rm -v *.o *.bin *.so
rm *~ # Remove temporary files 
\end{verbatim}

\hypertarget{inux15fa-adux131mlarux131}{%
\subsection{İnşa Adımları}\label{inux15fa-adux131mlarux131}}

ADIM 1: Geliştirici tarafından gerçekleştirilen kısım

\begin{itemize}
\tightlist
\item
  GNU autotools araçlarının gerektirdiği ek meta veri dosyaları oluşturun
\end{itemize}

\texttt{\$\ touch\ README\ NEWS\ AUTHORS\ ChangeLog\ \#\ Set\ metadata\ files}

ADIM 2: Geliştirici tarafından gerçekleştirilen kısım

\begin{itemize}
\tightlist
\item
  ./Configure kabuk komut dosyası ve ek dosyaları oluşturmak için atuoreconf komutunu çalıştırın.
\item
  Çalıştır: \texttt{\$\ autoreconf\ -i\ -v}
\end{itemize}

\begin{verbatim}
$ autoreconf -i -v 

autoreconf: Entering directory `.'
autoreconf: configure.ac: not using Gettext
autoreconf: running: aclocal 
autoreconf: configure.ac: tracing
autoreconf: configure.ac: not using Libtool
autoreconf: running: /usr/bin/autoconf
autoreconf: configure.ac: not using Autoheader
autoreconf: running: automake --add-missing --copy --no-force
configure.ac:11: warning: unknown warning category 'extra'
configure.ac:17: installing './compile'
configure.ac:11: installing './install-sh'
configure.ac:11: installing './missing'
Makefile.am: installing './depcomp'
autoreconf: Leaving directory `.'
\end{verbatim}

ADIM 3: Kullanıcı tarafından gerçekleştirilen kısım

\begin{itemize}
\tightlist
\item
  Aşağıdaki adımlar, uygulamayı sisteme veya yerel bir dizine yüklemek için son kullanıcılar veya bakımcılar tarafından gerçekleştirilir.
\item
  Yapılandırma komut dosyasını çalıştırarak Makefile dosyasını oluşturun.
\end{itemize}

Çalıştır: \texttt{\$\ ./configure}

\begin{verbatim}
$ ./configure 
checking for a BSD-compatible install... /usr/bin/install -c
checking whether build environment is sane... yes
checking for a thread-safe mkdir -p... /usr/bin/mkdir -p
checking for gawk... gawk

... ... ...   ... ... ...   ... ... ...   ... ... ... 
     # Suppress output for breviety purposes 
... ... ...   ... ... ...   ... ... ...   ... ... ... 

checking dependency style of g++... gcc3
checking that generated files are newer than configure... done
configure: creating ./config.status
config.status: creating Makefile
config.status: executing depfiles commands
\end{verbatim}

Çalıştır: Clang kullanmak için \$ ./configure CC=clang CXX=clang++

\texttt{\$\ ./configure\ CC=clang\ CXX=clang++}

ADIM 4: Kullanıcı tarafından gerçekleştirilen kısım

\begin{itemize}
\tightlist
\item
  Makefile'ı çalıştırın (GNU make)
\end{itemize}

Çalıştır: \texttt{\$\ make}

\begin{verbatim}
$ make

g++ -DPACKAGE_NAME=\"app\" -DPACKAGE_TARNAME=\"app\" -DPACKAGE_VERSION=\"0.1\"  \
    -DPACKAGE_STRING=\"app\ 0.1\" -DPACKAGE_BUGREPORT=\"\" -DPACKAGE_URL=\"\"   \
     -DPACKAGE=\"app\" -DVERSION=\"0.1\" -I.     -g -O2 -MT app.o -MD -MP -MF   \
    .deps/app.Tpo -c -o app.o app.cpp \
    mv -f .deps/app.Tpo .deps/app.Po \

 g++  -g -O2   -o app app.o  
\end{verbatim}

Çalıştır: \texttt{\$\ make\ install}

\begin{verbatim}
$ make install
make[1]: Entering directory '/home/archbox/projects/autools1'
 /usr/bin/mkdir -p '/usr/local/bin'
  /usr/bin/install -c app '/usr/local/bin'
/usr/bin/install: cannot create regular file '/usr/local/bin/app': Permission denied
make[1]: *** [Makefile:313: install-binPROGRAMS] Error 1
make[1]: Leaving directory '/home/archbox/projects/autools1'
make: *** [Makefile:615: install-am] Error 2
\end{verbatim}

Çalıştır: \texttt{\$\ make\ dist} =\textgreater{} Bir tarball dağılımı oluşturun.

\begin{verbatim}
$ make dist
make  dist-gzip am__post_remove_distdir='@:'
make[1]: Entering directory '/home/archbox/projects/autools1'
if test -d "app-0.1"; then find "app-0.1" -type d ! -perm -200 -exec chmod u+w {} ... ... 
test -d "app-0.1" || mkdir "app-0.1"

  .. ... ... ... ... ... ... 
rm -rf "app-0.1" || { sleep 5 && rm -rf "app-0.1"; }; else :; fi
\end{verbatim}

Tarball dosyasını kontrol edin:

\begin{verbatim}
    $ tar -tzvf app-0.1.tar.gz 
drwxrwxr-x 1000/1000         0 2019-10-21 15:23 app-0.1/
-rw-rw-r-- 1000/1000     42147 2019-10-21 15:22 app-0.1/aclocal.m4
-rwxrwxr-x 1000/1000    154043 2019-10-21 15:22 app-0.1/configure
-rw-rw-r-- 1000/1000       546 2019-10-21 15:01 app-0.1/configure.ac
-rw-rw-r-- 1000/1000       101 2019-10-21 14:28 app-0.1/app.cpp
-rwxr-xr-x 1000/1000     14676 2019-10-21 15:22 app-0.1/install-sh
-rw-rw-r-- 1000/1000         0 2019-10-21 15:02 app-0.1/ChangeLog
-rwxr-xr-x 1000/1000      6872 2019-10-21 15:22 app-0.1/missing
-rwxr-xr-x 1000/1000     23566 2019-10-21 15:22 app-0.1/depcomp
-rw-rw-r-- 1000/1000         0 2019-10-21 15:02 app-0.1/AUTHORS
-rw-r--r-- 1000/1000     35147 2019-10-21 15:01 app-0.1/COPYING
-rw-r--r-- 1000/1000     15756 2019-10-21 15:01 app-0.1/INSTALL
-rwxr-xr-x 1000/1000      7381 2019-10-21 15:22 app-0.1/compile
-rw-rw-r-- 1000/1000       117 2019-10-21 14:52 app-0.1/Makefile.am
-rw-rw-r-- 1000/1000         0 2019-10-21 15:02 app-0.1/README
-rw-rw-r-- 1000/1000         0 2019-10-21 15:02 app-0.1/NEWS
-rw-rw-r-- 1000/1000     23477 2019-10-21 15:22 app-0.1/Makefile.in
\end{verbatim}

ADIM 5:

\begin{itemize}
\tightlist
\item
  Uygulamayı test edin.
\end{itemize}

\begin{verbatim}
# Check executable file 
$ file app
app: ELF 64-bit LSB executable, x86-64, version 1 (SYSV), dynamically linked, interpreter 
/lib64/l, for GNU/Linux 3.2.0, BuildID[sha1]=29f5b104ac648c3286ed616ea4cf4007a6b51ae2, 
with debug_info, not stripped

# Run native executable 
$ ./app
Hello world autotools
\end{verbatim}

\hypertarget{uygulamayux131-yeni-bir-konuma-oluux15fturun-ve-yuxfckleyin}{%
\subsubsection{Uygulamayı yeni bir konuma oluşturun ve yükleyin}\label{uygulamayux131-yeni-bir-konuma-oluux15fturun-ve-yuxfckleyin}}

Configure betiğini çalıştırın.

\begin{verbatim}
# Build with clang and install at ~/opt/app-test 
$ ./configure CC=clang CXX=clang++ --prefix=$HOME/opt/app-test
\end{verbatim}

Make ve make install komutlarını çalıştırın

\begin{verbatim}
$ make install
make[1]: Entering directory '/home/archbox/projects/autools1'
 /usr/bin/mkdir -p '/home/archbox/opt/app-test/bin'
  /usr/bin/install -c app '/home/archbox/opt/app-test/bin'
make[1]: Nothing to be done for 'install-data-am'.
make[1]: Leaving directory '/home/archbox/projects/autools1'
\end{verbatim}

Test kurulumu:

\begin{verbatim}
$ tree /home/archbox/opt/app-test
/home/archbox/opt/app-test
└── bin
    └── app

1 directory, 1 file
\end{verbatim}

Uygulamayı çalıştırın:

\begin{verbatim}
$ /home/archbox/opt/app-test/bin/app 
Hello world autotools
\end{verbatim}

\hypertarget{ileri-duxfczey-tavsiye-kaynaklar}{%
\section{İleri düzey tavsiye kaynaklar}\label{ileri-duxfczey-tavsiye-kaynaklar}}

\begin{itemize}
\tightlist
\item
  \href{https://eklitzke.org/how-to-autotools}{Autotools Nasıl Kullanılır}
\item
  \href{https://www.star.bnl.gov/~liuzx/autobook.html}{GNU Autotools Kitabı}
\item
  \href{https://elinux.org/images/4/43/Petazzoni.pdf}{Gömülü Linux Konferansı 2016 - GNU Autotools Eğitimi}
\item
  \href{http://freesoftwaremagazine.com/articles/brief_introduction_to_gnu_autotools/}{Bölüm 1: GNU Otomobillerine Kısa Bir Giriş}
\item
  \href{https://stackoverflow.com/questions/10999549/how-do-i-create-a-configure-script}{Nasıl yapılandırma betiği oluştururum?}
\item
  \href{https://developer.fedoraproject.org/tech/languages/c/autotools.html}{Autotools --- Fedora Geliştirici Portalı}
\item
  \href{http://thegreyblog.blogspot.com/2013/09/cc-project-built-with-gnu-build-system.html}{GNU Derleme Sistemi (A.K.A. GNU Autotools) ile Üretilen C / C ++ Projesi: NetBeans vs.~Eclipse CDT}
\end{itemize}

\hypertarget{msbuild}{%
\chapter{Msbuild}\label{msbuild}}

\hypertarget{msbuild-nedir}{%
\section{Msbuild Nedir}\label{msbuild-nedir}}

Msbuild kullanımı avantajı dezavantajı\ldots{}

\hypertarget{msbuild-kullanux131mux131}{%
\section{Msbuild Kullanımı}\label{msbuild-kullanux131mux131}}

Msbuild kullanımı avantajı dezavantajı\ldots{}

\hypertarget{msbuild-avantajlarux131}{%
\section{Msbuild Avantajları}\label{msbuild-avantajlarux131}}

Msbuild kullanımı avantajı dezavantajı\ldots{}

\hypertarget{msbuild-dezavantajlarux131}{%
\section{Msbuild Dezavantajları}\label{msbuild-dezavantajlarux131}}

Msbuild kullanımı avantajı dezavantajı\ldots{}

\hypertarget{meson}{%
\chapter{Meson}\label{meson}}

\hypertarget{meson-nedir}{%
\section{Meson Nedir}\label{meson-nedir}}

Meson kullanımı avantajı dezavantajı\ldots{}

\hypertarget{meson-kullanux131mux131}{%
\section{Meson Kullanımı}\label{meson-kullanux131mux131}}

Meson kullanımı avantajı dezavantajı\ldots{}

\hypertarget{meson-avantajlarux131}{%
\section{Meson Avantajları}\label{meson-avantajlarux131}}

Meson kullanımı avantajı dezavantajı\ldots{}

\hypertarget{meson-dezavantajlarux131}{%
\section{Meson Dezavantajları}\label{meson-dezavantajlarux131}}

Meson kullanımı avantajı dezavantajı\ldots{}

\hypertarget{qmake}{%
\chapter{Qmake}\label{qmake}}

\hypertarget{qmake-nedir}{%
\section{Qmake Nedir}\label{qmake-nedir}}

Qmake kullanımı avantajı dezavantajı\ldots{}

\hypertarget{qmake-kullanux131mux131}{%
\section{Qmake Kullanımı}\label{qmake-kullanux131mux131}}

Qmake kullanımı avantajı dezavantajı\ldots{}

\hypertarget{qmake-avantajlarux131}{%
\section{Qmake Avantajları}\label{qmake-avantajlarux131}}

Qmake kullanımı avantajı dezavantajı\ldots{}

\hypertarget{qmake-dezavantajlarux131}{%
\section{Qmake Dezavantajları}\label{qmake-dezavantajlarux131}}

Qmake kullanımı avantajı dezavantajı\ldots{}

\hypertarget{kbuild}{%
\chapter{Kbuild}\label{kbuild}}

\hypertarget{kbuild-nedir}{%
\section{Kbuild Nedir}\label{kbuild-nedir}}

Kbuild kullanımı avantajı dezavantajı\ldots{}

\hypertarget{kbuild-kullanux131mux131}{%
\section{Kbuild Kullanımı}\label{kbuild-kullanux131mux131}}

Kbuild kullanımı avantajı dezavantajı\ldots{}

\hypertarget{kbuild-avantajlarux131}{%
\section{Kbuild Avantajları}\label{kbuild-avantajlarux131}}

Kbuild kullanımı avantajı dezavantajı\ldots{}

\hypertarget{kbuild-dezavantajlarux131}{%
\section{Kbuild Dezavantajları}\label{kbuild-dezavantajlarux131}}

Kbuild kullanımı avantajı dezavantajı\ldots{}

\hypertarget{build2}{%
\chapter{Build2}\label{build2}}

\hypertarget{build2-nedir}{%
\section{Build2 Nedir}\label{build2-nedir}}

Build2 kullanımı avantajı dezavantajı\ldots{}

\hypertarget{build2-kullanux131mux131}{%
\section{Build2 Kullanımı}\label{build2-kullanux131mux131}}

Build2 kullanımı avantajı dezavantajı\ldots{}

\hypertarget{build2-avantajlarux131}{%
\section{Build2 Avantajları}\label{build2-avantajlarux131}}

Build2 kullanımı avantajı dezavantajı\ldots{}

\hypertarget{build2-dezavantajlarux131}{%
\section{Build2 Dezavantajları}\label{build2-dezavantajlarux131}}

Build2 kullanımı avantajı dezavantajı\ldots{}

\hypertarget{xmake}{%
\chapter{Xmake}\label{xmake}}

\hypertarget{xmake-nedir}{%
\section{Xmake Nedir}\label{xmake-nedir}}

Xmake kullanımı avantajı dezavantajı\ldots{}

\hypertarget{xmake-kullanux131mux131}{%
\section{Xmake Kullanımı}\label{xmake-kullanux131mux131}}

Xmake kullanımı avantajı dezavantajı\ldots{}

\hypertarget{xmake-avantajlarux131}{%
\section{Xmake Avantajları}\label{xmake-avantajlarux131}}

Xmake kullanımı avantajı dezavantajı\ldots{}

\hypertarget{xmake-dezavantajlarux131}{%
\section{Xmake Dezavantajları}\label{xmake-dezavantajlarux131}}

Xmake kullanımı avantajı dezavantajı\ldots{}

\hypertarget{bazel}{%
\chapter{Bazel}\label{bazel}}

\hypertarget{bazel-nedir}{%
\section{Bazel Nedir}\label{bazel-nedir}}

Bazel kullanımı avantajı dezavantajı\ldots{}

\hypertarget{bazel-kullanux131mux131}{%
\section{Bazel Kullanımı}\label{bazel-kullanux131mux131}}

Bazel kullanımı avantajı dezavantajı\ldots{}

\hypertarget{bazel-avantajlarux131}{%
\section{Bazel Avantajları}\label{bazel-avantajlarux131}}

Bazel kullanımı avantajı dezavantajı\ldots{}

\hypertarget{bazel-dezavantajlarux131}{%
\section{Bazel Dezavantajları}\label{bazel-dezavantajlarux131}}

Bazel kullanımı avantajı dezavantajı\ldots{}

\hypertarget{part-paket-yuxf6neticileri}{%
\chapter{(PART) Paket Yöneticileri}\label{part-paket-yuxf6neticileri}}

\hypertarget{conan}{%
\chapter{Conan}\label{conan}}

\hypertarget{conan-buxf6luxfcm-iuxe7eriux11fi}{%
\section{Conan Bölüm içeriği}\label{conan-buxf6luxfcm-iuxe7eriux11fi}}

Conan kullanımı avantajı dezavantajı\ldots{}

\end{document}
